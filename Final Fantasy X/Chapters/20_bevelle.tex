\chapter{Bevelle}
\begin{enumerate}
    \item Use a Mega-Potion
    \ifthenelse{\equal{\blitzresult}{win}}{}{
        \ifthenelse{\equal{\blitzresult}{both}}{
            \item \textit{If you \lostblitz:}
    }{}
        \begin{equip}
            \begin{itemize}
                \tidusf Equip Sonic Steel
            \end{itemize}
        \end{equip}
    }
    \item \textit{With Sleeping Powder:}
\end{enumerate}
\begin{battle}{Guard Fights - Sleeping Powder}
    \begin{itemize}
        \item \textit{Fights 1 and 3 (3 Monks):}
        \begin{itemize}
            \tidusf Attack
            \item Others: Defend or use Distillers
        \end{itemize}
        \item \textit{Fights 2 and 4 (2 Monks and a YKT-63):}
        \begin{itemize}
            \tidusf Attack the YKT-63
            \rikkuf Sleeping Powder
            \kimahrif Smoke Bomb/Silence Grenade/Sleeping Powder
            \blitzballdetermination[true]{}{%
                \item \textit{If the YKT-63 is still alive} Use a Lightning Marble/Arctic Wind/Fish Scale or Attack with \tidus
            }
        \end{itemize}
        \item \textit{Fight 5 (2 Monks and a YAT-99):}
        \begin{itemize}
            \item \textit{If you have 2 Smoke Bombs/Sleeping Powders/Silence Grenades:}
            \begin{itemize}
                \tidusf Haste \rikku
                \rikkuf Sleeping Powder/Smoke Bomb/Silence Grenade
                \rikkuf If the Guards are sleeping use a Bomb Core on the YAT-99
                \rikkuf Sleeping Powder/Smoke Bomb/Silence Grenade
                \tidusf Attack
            \end{itemize}
            \item \textit{If you have 2 Bomb Cores:}
            \begin{itemize}
                \tidusf Attack the Monks
                \item Others: Use Bomb Core x2 on the YAT-99
            \end{itemize}
        \end{itemize}
    \end{itemize}
\end{battle}
\bothvfill
\lossvfill
\winvfill
\ 
\bothcb
\wincb
\losscb
\ 
\ \bothnewline \winnewline \lossnewline
\begin{enumerate}[resume]
    \item \textit{Without Sleeping Powder:}
    \begin{itemize}
        \item Keep \formation{\tidus}{\rikku}{\lulu} for the first 4 fights, \formation{\tidus}{\rikku}{\kimahri} for the last one
    \end{itemize}
\end{enumerate}
\begin{battle}{Guard Fights - No Sleeping Powder}
    \begin{itemize}
        \item \textit{Fights 1 and 3 (3 Monks):}
        \begin{itemize}
            \tidusf Attack
            \item Others: Defend or use Distillers
        \end{itemize}
        \item \textit{Fights 2 and 4 (2 Monks and a YKT-63):}
        \begin{itemize}
            \switch{\tidus}{\kimahri}
            \kimahrif Silence Grenade/Smoke Bomb
            \rikkuf Silence Grenade/Smoke Bomb
            \switch{\kimahri}{\tidus}
            \tidusf Attack the YKT-63
            \blitzballdetermination[true]{}{%
                \item \textit{If the YKT-63 is still alive} Use a Lightning Marble/Arctic Wind/Fish Scale or Attack with \tidus
            }
        \end{itemize}
        \item \textit{Fight 5 (2 Monks and a YAT-99):}
        \begin{itemize}
            \item \textit{If you have 2 Smoke Bombs/Silence Grenades:}
            \begin{itemize}
                \tidusf Haste \rikku
                \rikkuf Smoke Bomb/Silence Grenade x2
                \tidusf Attack
            \end{itemize}
            \item \textit{If you have 2 Bomb Cores:}
            \begin{itemize}
                \tidusf Attack the Monks
                \item Others: Use Bomb Core x2 on the YAT-99
            \end{itemize}
        \end{itemize}
    \end{itemize}
\end{battle}
\begin{enumerate}[resume]
    \item \sd, \skippablefmv[1:30], \sd\ on \yuna\ dialogue. \skippablefmv[30], \sd. Use lift, \sd.
\end{enumerate}
\bothvfill
\winvfill
\lossvfill
\ 
\begin{trial}
    \begin{itemize}
        \item \textit{Upper section:}
        \begin{itemize}
            \item Push the pedestal in
            \item Press X
            \item Go left at the 2nd junction
            \item Take sphere, push pedestal back
            \item At the 3rd junction, go back (hold X)
            \item Go left at the 2nd junction
            \item Place sphere into wall, push pedestal back
            \item At the 3rd junction, go back (hold X)
            \item Go left at the 1st junction (hold X after the 2nd junction)
        \end{itemize}
        \item \textit{Lower section (1st visit):}
        \begin{itemize}
            \item The platform will automatically stop at the 1st junction
            \item After the platform stops, press X the 2nd time the arrow is pointing left
            \item Go right at the 3rd junction (hold X after the 2nd junction)
            \item Take Glyph sphere from wall, push pedestal back
            \item At the 4th junction go right (hold X)
            \item Place Glyph sphere into pedestal
            \item Take Bevelle sphere from pedestal
            \item Place Bevelle sphere into the wall
            \item Take the Glyph sphere from pedestal
            \item Place Glyph sphere into the next wall
            \item Take Destruction sphere from the new wall
            \item Place Destruction sphere on the pedestal
            \item Take Bevelle sphere from the wall
            \item Push pedestal back and fall off the edge
        \end{itemize}
        \item \textit{Lower section (2nd visit):}
        \begin{itemize}
            \item Go straight (start holding X before the platform stops)
            \item At the 3rd junction go right (hold X after the 2nd junction)
            \item Place Bevelle sphere on the pedestal
            \item Take Destruction sphere from the pedestal
            \item Place Destruction sphere into wall
            \item Push pedestal back and fall off the edge
        \end{itemize}
        \item \textit{Lower section (3nd visit):}
        \begin{itemize}
            \item Go straight
            \item At the 2nd junction go right (hold X)
            \item Push pedestal
            \item Go up the stairs, open the chest
        \end{itemize}
    \end{itemize}
\end{trial}
\begin{enumerate}[resume]
    \item \sd, name \bahamut, don't save, \sd
\end{enumerate}
\lossvfill
\ 
\losscb
\ \lossnewline \ 
