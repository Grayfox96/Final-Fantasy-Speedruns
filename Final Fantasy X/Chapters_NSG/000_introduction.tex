\chapter{Introduction}

\textbf{\Large Some beginning information about the run:}

\vspace{\baselineskip}

\begin{itemize}
\item You should be able to complete the first run that you do, as long as you follow the notes exactly. Misreading them can lead to runs that cannot complete. Don't try to do something else because you think it will also work, unless you've tried it before. Information about WHY we do these things are not present in these notes, as they are outside the scope of this document, but if you ask someone will definitely be able to tell you.
\end{itemize}

\vspace{\baselineskip}

\textbf{\Large Some information about how these notes are laid out:}

\vspace{\baselineskip}

\begin{itemize}
\liteversiondetermination{Exclude}{%
\item There are a few acronyms used throughout the run.
\begin{itemize}
	\item \sd: \textbf{Skip Dialogue}. During some cutscenes, some of the dialogue is skippable. As soon as the text finishes appearing on the screen, you can hit \textbf{Confirm} to cause it to disappear. This will stop the Voice Over lines from completing, causing the cutscene to progress faster. As a result, you can mash during this to progress faster.
	\item \cs: \textbf{Cutscene}. In game rendered cutscene. Can't do anything about it, just take a break. Usually they will have the approximate time that the cutscenes take, so you can plan your breaks better. These are timed for PS2.
	\item \fmv: \text{Full Motion Video}. Pre-rendered cutscene. Can't do anything about it (usually), just take a break. Usually they will have the approximate time that the cutscenes take, so you can plan your breaks better. These are timed for PS2.
	\item \skippablefmv: \textbf{Skippable Full Motion Video}. Pre-rendered cutscene, but you can skip these if you are on PC. They still have times, because these are not skippable on PS2.
	\item \save: Touching Save Spheres will full heal you. Touch the save sphere, and then cancel out.
\end{itemize}
\vspace{\baselineskip}
}

\item Read each page as such: Left column, then right column, then the next page. There are some instances Read the columns left column first, then right column, then next page. There are some instances where there will be an instruction box that takes up both columns - in this case, do whatever is above the instruction box first (left column, then right column), then do whatever is below the instruction box the same way (left column, then right column)

\vspace{\baselineskip}

\item Each bullet point is their own item. Do what it says there before going to the next one.

\vspace{\baselineskip}

\item There are instances where you have to get an item, or overdrive, etc before progressing. If the notes say to do so... \textbf{Do So}. These notes don't contain many backup strats.
\end{itemize}

\vspace{\baselineskip}

\textbf{\Large Some important considerations for the run:}

\vspace{\baselineskip}

\textbf{\large Encounter Count:}

One of the mechanics present in the game is that Aeons get more powerful the more encounters have happened in the game. Starting at 60 encounters and every 30 encounters thereafter the base power of the Aeons increases. In this route we repeatedly force fights against Lord Ochu in Kilika and Flee to increase the encounter count in the fastest way possible. This route forces 140 Lord Ochu Encounters. On PC this route saves time vs the existing route which only forces 50 Lord Ochu Encounters, however on PS2 it is slower due to the increased load times of encounters.

\vspace{\baselineskip}

\textbf{\large Tidus' Overdrive:}

\vspace{\baselineskip}

By the late game we need to have Tidus learn his Slice and Dice overdrive ability which becomes available after using 10 total overdrives throughout the run. To that end it is a good idea to try to fit in a Spiral Cut (Tidus' starting overdrive) wherever possible. The route has 3 mandatory Spiral Cut uses on Sinspawn Ammes, Sinspawn Echuilles and Seymour. Others Good Locations to fit in a Spiral Cut are:

\vspace{\baselineskip}

\begin{itemize}

	\item Garuda (Luca)
	\item Extractor
	\item Spherimorph (First Turn)
	\item Defender X

\end{itemize}

\vspace{\baselineskip}
	
\textbf{\large Lulu's Overdrive:}

\vspace{\baselineskip}

In the late game we utilise Lulu's Fury Overdrive in combination with Rikku's Trio of 9999 Mix to deal large damage to enemies. This overdrive works by rotating the right thumbstick with the number of hits increaseing with more rotations. There are 3 Bosses in the game that have mandatory 7 hit requirements on Fury to be able to defeat the Boss. Some people find this difficult to achieve consistently so you should practice this overdrive before your first run and be confident in getting at least 7 hits. The following resources may also be of help:

\vspace{\baselineskip}

\begin{itemize}

	\item Fury Tips by CrimsonInferno9 - \url{https://youtu.be/CuwgUXgQMHc}

\end{itemize}

\newpage